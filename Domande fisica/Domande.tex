\documentclass[12pt,a4paper]{article}
\usepackage[utf8]{inputenc}
\usepackage{amsmath}
\usepackage{amsfonts}
\usepackage{nicefrac}
\usepackage{amssymb}
\usepackage{amsthm}
\usepackage[pdftex, pdfborderstyle={/S/U/W 0}]{hyperref}
\numberwithin{equation}{subsection}
\usepackage{commath}
\usepackage[italian]{babel}
\usepackage{indentfirst}
\newcommand*{\defeq}{\stackrel{\text{def}}{=}}
\author{Jacopo Tissino}
\title{Matematica}
\begin{document}

Ho trovato le equazioni di Navier-Stokes per un fluido non-comprimibile formulate come:

\begin{equation}
\frac{\partial \mathbf{v}}{\partial t} + ( \mathbf{v} \nabla ) \mathbf{v}  = -\frac{\nabla p}{\rho} + \nu \nabla^2 \mathbf{v}
\end{equation}
\begin{equation}
\nabla \mathbf{v} = 0
\end{equation}

Dove $\nu$ è la viscosità, e immagino $\mathbf{v}$ sia la velocità, $p$ la pressione, $\rho$ la densità. Torna tutto a livello dimensionale, ma non capisco il significato di $( \mathbf{v} \nabla ) \mathbf{v}$: è diverso scrivere $ \mathbf{v} \nabla $ rispetto a $ \nabla \mathbf{v} $? Rappresenta il gradiente o altre operazioni, e se è il gradiente, come si fa a fare il gradiente di un campo vettoriale?

Poi, 

\end{document}