\documentclass[12pt,a4paper]{article}
\usepackage[utf8]{inputenc}
\usepackage{amsmath}
\usepackage{amsfonts}
\usepackage{nicefrac}
\usepackage[margin=2.5cm]{geometry}
\usepackage{amssymb}
\usepackage{amsthm}
\usepackage[pdftex, pdfborderstyle={/S/U/W 0}]{hyperref}
\numberwithin{equation}{subsection}
\usepackage{commath}
\usepackage[italian]{babel}
\usepackage{indentfirst}
\usepackage{url}
\newcommand*{\defeq}{\stackrel{\text{def}}{=}}
\author{Jacopo Tissino \\
VB
(CLIL)\\
Liceo Scientifico M. Grigoletti}
\date{A. S. 2015--16}
\title{\huge{\textbf{Turbolenza}}\\
\Large{Fluidodimamica e Van Gogh}}
\begin{document}

\maketitle

\begin{quote}
``When I meet God, I am going to ask him two questions: Why relativity? And why turbulence? I really believe he will have an answer for the first.''
\end{quote}

Questa citazione, attribuita tavolta a Werner Heisenberg \cite{heisenberg}, testimonia la complessità del fenomeno della turbolenza, ovvero del regime di flusso caotico che tipicamente si osserva in fluidi nei quali le forze inerziali dominano su quelle viscose. 
Nel 1941, Andrey Kolmogorov tentò di descrivere statisticamente la distribuzione dell'energia nello spettro delle lunghezze dei vortici. Seppur imperfetta, la sua teoria ritenne una grande importanza nello studio della turbolenza, e costituì un punto di partenza per le successive.

Alcuni pittori --- come Vincent van Gogh --- sono spesso considerati ``turbolenti'', ma solo tramite un'osservazione qualitativa. Un gruppo di scienziati spagnoli \cite{study2006} ha tentato di quantificare quest'intuizione, applicando le equazioni di Kolmogorov alla distribuzione della differenza di luminanza fra pixel a determinate distanze delle immagini digitalizzate dei quadri, con un risultato sorprendente: solo alcuni dei dipinti del pittore olandese, fra cui \emph{``Notte Stellata''}, rispondono alle distribuzioni di probabilità previste, mentre altri, come l'\emph{``Autoritratto con pipa e orecchio bendato''}, no: prima di dipinger-  lo, all'autore era stato somministrato brumuro di potassio come calmante, in seguito all'episodio psicotico che l'aveva appunto portato ad amputarsi parte dell'orecchio.

Si può discutere se questo risultato significhi --- come sostengono gli autori dello studio --- che Van Gogh sia riuscito a catturare l'``essenza'' della turbolenza, o se le proprietà rilevate siano segno di una diverso modo di dipingere, senza nessun collegamento con la fisica.
Comunque, la ricerca di \emph{pattern} nell'arte è sempre affascinante, e forse può fornire un nuovo punto di vista nella nostra interpretazione dell'artista.

\vspace{7mm}

%Nello studio della turbolenza si aggiunge al termine della viscosità cinematica una nuova ``viscosità turbolenta''. Parlandone con un amico, questi l'ha descritta come un ``vaso di Pandora'' poiché, malgrado sia facilmente verificabile che un fluido in regime di turbolenza presenti un comportamento equivalente ad un aumento della sua viscosità (unicamente fittizio poiché la viscosità è una proprietà fisica propria del fluido), voler calcolare effettivamente questo termine apre un mondo di diverse teorie e ipotesi.

Avendo già sentito parlare dei modelli statistici di fenomeni caotici, lo studio in questione \cite{study2006} è stato un felice ritrovamento, in quanto un altro mio interesse da tempo sono gli aspetti fisici e/o tecnici dell'arte.

Per me, ha più fascino l'espressione vincolata rispetto a quella completamente libera, e dunque è intrigante la possibilità che Van Gogh, nel ricercare emotività e bellezza nella creazione dei suoi quadri, le abbia espresse inconsciamente secondo una legge fisica della turbolenza, la quale dunque, a dispetto della sua ``fredda'' formulazione matematica, riesce a suscitare emozioni nella mente dell'uomo.

\begin{thebibliography}{1}

\bibitem{heisenberg}  
  \url{http://scienceworld.wolfram.com/biography/Heisenberg.html}

\bibitem{study2006}
  J.L. Aragón, Gerardo G. Naumis, M. Bai, M. Torres, P.K. Maini, \\
  \emph{Turbulent luminance in impassioned van Gogh paintings}, \\
  \url{http://arxiv.org/abs/physics/0606246}, 
  2006.

\bibitem{kolmogorovpres}
  Karima Khusnutdinova, 
  \emph{Kolmogorov's $5/3$ law}, \\
  \url{http://homepages.lboro.ac.uk/~makk/mathrev_kolmogorov.pdf},
  2009.
  
\bibitem{powerlaw}
  Terence Tao,
  \emph{Kolmogorov’s power law for turbulence}, \\
  \url{https://terrytao.wordpress.com/2014/05/15/kolmogorovs-power-law-for-turbulence/}, 2014.

\bibitem{gogh1}
  Marianne Freiberger, 
  \emph{Troubled minds and perfect turbulence}, \\
  \url{https://plus.maths.org/content/troubled-minds-and-perfect-turbulence},
  2006.
  
\bibitem{gogh2}
  Philip Ball, 
  \emph{Van Gogh painted perfect turbulence}, \\
  \url{http://www.nature.com/news/2006/060703/full/news060703-17.html#close},
  2006,
  su \emph{``Nature''}.


\bibitem{derivationns}
  \url{https://en.wikipedia.org/wiki/Derivation_of_the_Navier%E2%80%93Stokes_equations}
  
\bibitem{nseqs}
  \url{https://en.wikipedia.org/wiki/Navier–Stokes_equations}

\bibitem{tensorcalculus}
  \url{http://homepages.engineering.auckland.ac.nz/~pkel015/SolidMechanicsBooks/Part_III/Chapter_1_Vectors_Tensors/Vectors_Tensors_14_Tensor_Calculus.pdf}

\bibitem{turbulence2}
  \url{https://engineering.dartmouth.edu/~d30345d/books/EFM/chap8.pdf}

\bibitem{fluiddynamics}
  \url{https://www.materials.uoc.gr/el/grad/courses/METY101/FLUID_DYNAMICS_CRETE.pdf}     

\bibitem{sholar-turb}
  \url{http://www.scholarpedia.org/article/Turbulence}
  
\bibitem{viscousstresstensor}
  \url{https://en.wikipedia.org/wiki/Viscous_stress_tensor}

\bibitem{innerp}
  \url{https://en.wikipedia.org/wiki/Inner_product_space}
  
\bibitem{epsilon}  
  \url{http://www.cfd-online.com/Wiki/Turbulence_dissipation_rate}  
  
\bibitem{pdf}
  \url{https://en.wikipedia.org/wiki/Probability_density_function}
  
\bibitem{dispense}
  Valentino Pediroda, 
  \emph{Fluidodinamica} (dispense), A. A. 2005--2006, II semestre: capp. 4, 5, 7, 9.
  
\bibitem{engr}
  %Definizioni e cose sul prodotto interno.
  \url{https://www.engr.uky.edu/~acfd/lctr-notes634.pdf}  
  
\bibitem{scalingbuono}
  \url{http://research.me.udel.edu/~lwang/reprints/Wang_etal_JFM_1996.pdf}
  

\end{thebibliography}

\end{document}